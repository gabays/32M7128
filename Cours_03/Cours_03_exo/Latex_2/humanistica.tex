%This file derives from the ACL template: https://github.com/acl-org/acl-style-files. Adaptation in French: Simon Gabay

% This must be in the first 5 lines to tell arXiv to use pdfLaTeX, which is strongly recommended.
\pdfoutput=1
% In particular, the hyperref package requires pdfLaTeX in order to break URLs across lines.

\documentclass[11pt,french]{article}

% Remove the "review" option to generate the final version.
\usepackage[review]{humanistica}

% Standard package includes
\usepackage{times}
\usepackage{latexsym}

% For proper rendering and hyphenation of words containing Latin characters (including in bib files)
\usepackage[T1]{fontenc}
% For Vietnamese characters
% \usepackage[T5]{fontenc}
% See https://www.latex-project.org/help/documentation/encguide.pdf for other character sets

% This assumes your files are encoded as UTF8
\usepackage[utf8]{inputenc}

% This is not strictly necessary, and may be commented out,
% but it will improve the layout of the manuscript,
% and will typically save some space.
\usepackage{microtype}

%Translate to French
\usepackage{babel}
\def\frenchtablename{Tableau}

% If the title and author information does not fit in the area allocated, uncomment the following
%
%\setlength\titlebox{<dim>}
%
% and set <dim> to something 5cm or larger.

\title{Modèle pour Humanistica 2023}

% Author information can be set in various styles:
% For several authors from the same institution:
% \author{Author 1 \and ... \and Author n \\
%         Address line \\ ... \\ Address line}
% if the names do not fit well on one line use
%         Author 1 \\ {\bf Author 2} \\ ... \\ {\bf Author n} \\
% For authors from different institutions:
% \author{Author 1 \\ Address line \\  ... \\ Address line
%         \And  ... \And
%         Author n \\ Address line \\ ... \\ Address line}
% To start a seperate ``row'' of authors use \AND, as in
% \author{Author 1 \\ Address line \\  ... \\ Address line
%         \AND
%         Author 2 \\ Address line \\ ... \\ Address line \And
%         Author 3 \\ Address line \\ ... \\ Address line}

\author{First Author \\
  Affiliation / Adresse ligne 1 \\
  Affiliation / Adresse ligne 2 \\
  Affiliation / Adresse ligne 3 \\
  \texttt{email@domaine} \\\And
  Second Author \\
  Affiliation / Adresse ligne 1 \\
  Affiliation / Adresse ligne 2 \\
  Affiliation / Adresse ligne 3 \\
  \texttt{email@domaine} \\}

\begin{document}
\maketitle
\begin{abstract}
Ce document est un complément aux directives générales pour les auteurs Humanistica. Il contient des instructions sur l'utilisation du modèle Microsoft Word pour la conférence \LaTeX{}. Le document lui-même est conforme à ses propres spécifications et est donc un exemple de ce à quoi votre article devrait ressembler. Ces instructions doivent être utilisées à la fois pour les articles soumis pour examen et pour les versions finales des articles acceptés.
\end{abstract}

\section{Introduction}

Ces instructions sont destinées aux auteur.trice.s qui soumettent des articles aux conférences Humanistica en utilisant \LaTeX. Tou.te.s les auteur.trice.s doivent suivre les instructions générales pour Humanistica, et ce document contient des instructions supplémentaires pour le style \LaTeX{} des dossiers.

Les modèles incluent la source \LaTeX{} de ce document (\texttt{Humanistica.tex}), le fichier de style \LaTeX{} utilisé pour le formater (\texttt{humanistica.sty}),
un style de bibliographie ACL (\texttt{acl\_natbib.bst}),
un exemple de bibliographie (\texttt{bibliographie.bib}).

\section{Moteurs}
Pour produire un fichier PDF, pdf\LaTeX{} est fortement recommandé (plutôt que \LaTeX{} d'origine plus dvips+ps2pdf ou dvipdf). Xe\LaTeX{} produit également des fichiers PDF et est particulièrement adapté au texte dans des scripts non latins.

\section{Préambule}

La première ligne du fichier doit être
\begin{quote}
\begin{verbatim}
\documentclass[11pt]{article}
\end{verbatim}
\end{quote}

Pour charger le fichier de style dans la version de révision :
\begin{quote}
\begin{verbatim}
\usepackage[review]{humanistica}
\end{verbatim}
\end{quote}
Pour la version finale, omettez le \verb|review| option:
\begin{quote}
\begin{verbatim}
\usepackage{humanistica}
\end{verbatim}
\end{quote}

Pour utiliser Times Roman, mettez ce qui suit dans le préambule :
\begin{quote}
\begin{verbatim}
\usepackage{times}
\end{verbatim}
\end{quote}
(Des alternatives comme txfonts ou newtx sont également acceptables.)

Veuillez consulter la source \LaTeX{} de ce document pour des commentaires sur d'autres packages qui pourraient être utiles.

Définissez le titre et l'auteur en utilisant \verb|\title| et \verb|\auteur|. Dans la liste des auteurs, formatez plusieurs auteurs en utilisant \verb|\and| et \verb|\Et| et \verb|\AND| ; veuillez consulter la source \LaTeX{} pour des exemples.

Par défaut, la case contenant le titre et les noms des auteurs est réglée au minimum de 5 cm. Si vous avez besoin de plus d'espace, incluez ce qui suit dans le préambule :
\begin{quote}
\begin{verbatim}
\setlength\titlebox{<dim>}
\end{verbatim}
\end{quote}
où \verb|<dim>| est remplacé par une longueur. Ne réglez pas cette longueur à moins de 5 cm.

\section{Corps du texte}

\subsection{Notes de bas de page}

Les notes de bas de page sont insérées avec la commande \verb|\footnote| .\footnote{Ceci est une note de bas de page.}

\subsection{Tableaux and figures}

Voir Table~\ref{tab:accents} pour un exemple de tableau et sa légende.
\textbf{Ne remplacez pas les tailles de légende par défaut.}

\begin{table}
\centering
\begin{tabular}{lc}
\hline
\textbf{Commande} & \textbf{Sortie}\\
\hline
\verb|{\"a}| & {\"a} \\
\verb|{\^e}| & {\^e} \\
\verb|{\`i}| & {\`i} \\ 
\verb|{\.I}| & {\.I} \\ 
\verb|{\o}| & {\o} \\
\verb|{\'u}| & {\'u}  \\ 
\verb|{\aa}| & {\aa}  \\\hline
\end{tabular}
\begin{tabular}{lc}
\hline
\textbf{Commande} & \textbf{Sortie}\\
\hline
\verb|{\c c}| & {\c c} \\ 
\verb|{\u g}| & {\u g} \\ 
\verb|{\l}| & {\l} \\ 
\verb|{\~n}| & {\~n} \\ 
\verb|{\H o}| & {\H o} \\ 
\verb|{\v r}| & {\v r} \\ 
\verb|{\ss}| & {\ss} \\
\hline
\end{tabular}
\caption{Exemples de commandes pour les caractères accentués, à utiliser dans, par exemple, les entrées Bib\TeX{}.}
\label{tab:accents}
\end{table}

\subsection{Hyperliens}

Les utilisateurs d'anciennes versions de \LaTeX{} peuvent rencontrer l'erreur suivante lors de la compilation :
\begin{quote}
\tt\verb|\pdfendlink| ended up in different nesting level than \verb|\pdfstartlink|.
\end{quote}
Cela se produit lorsque pdf\LaTeX{} est utilisé et qu'une citation se divise sur une limite de page. La meilleure façon de résoudre ce problème est de mettre à niveau \LaTeX{} vers 2018-12-01 ou version ultérieure.

\subsection{Citations}

\begin{table*}
\centering
\begin{tabular}{lll}
\hline
\textbf{Sortie} & \textbf{commande natbib } & \textbf{Commande ancien style ACL}\\
\hline
\citep{Gusfield:97} & \verb|\citep| & \verb|\cite| \\
\citealp{Gusfield:97} & \verb|\citealp| & pas d'équivalent \\
\citet{Gusfield:97} & \verb|\citet| & \verb|\newcite| \\
\citeyearpar{Gusfield:97} & \verb|\citeyearpar| & \verb|\shortcite| \\
\hline
\end{tabular}
\caption{\label{citation-guide}
Commandes de citation prises en charge par le fichier de style.
Le style est basé sur le package natbib et prend en charge toutes les commandes de citation natbib.
Il prend également en charge les commandes définies dans les fichiers de style Humanistica précédents pour la compatibilité.
}
\end{table*}

Le tableau~\ref{citation-guide} montre la syntaxe prise en charge par les fichiers de style.
Nous vous encourageons à utiliser les styles natbib.
Vous pouvez utiliser la commande \verb|\citet| (citer dans le texte) pour obtenir des citations ``auteur (année)'', comme cette citation à un article de \citet{Gusfield:97}.
Vous pouvez utiliser la commande \verb|\citep| (citer entre parenthèses) pour obtenir les citations ``(auteur, année)'' \citep{Gusfield:97}.
Vous pouvez utiliser la commande \verb|\citealp| (citation alternative sans parenthèses) pour obtenir les citations ``auteur, année'', ce qui est pratique pour utiliser des citations entre parenthèses (par exemple \citealp{Gusfield:97}).

\subsection{References}

\nocite{Ando2005,andrew2007scalable,rasooli-tetrault-2015}

Les fichiers de style \LaTeX{} et Bib\TeX{} fournis suivent approximativement le format de l'American Psychological Association.
Si votre propre fichier bib s'appelle \texttt{bibliographie.bib}, placer ce qui suit avant toute annexe dans votre fichier \LaTeX{} générera la section des références pour vous :
\begin{quote}
\begin{verbatim}
\bibliography{bibliographie}
\end{verbatim}
\end{quote}

Veuillez consulter la section ~\ref{sec:bibtex} pour plus d'informations sur la préparation des fichiers Bib\TeX{}.

\subsection{Annexes}

Utilisez \verb|\appendice| avant toute section d'annexe pour changer la numérotation des sections en lettres. Voir Annexe~\ref{sec:appendix} pour un exemple.

\section{Bib\TeX{} Files}
\label{sec:bibtex}

Unicode ne peut pas être utilisé dans les entrées Bib\TeX{}, et certaines manières de saisir des caractères spéciaux peuvent perturber l'alphabétisation de Bib\TeX. La manière recommandée de saisir les caractères spéciaux est indiquée dans le tableau ~\ref{tab:accents}.

Veuillez vous assurer que les enregistrements Bib\TeX{} contiennent des DOI ou des URL lorsque cela est possibl.
Utilisez le champ \verb|doi| pour les DOI et le champ \verb|url|  pour les URL.
Si une entrée Bib\TeX{} a un champ URL ou DOI, le titre de l'article dans la section des références apparaîtra comme un lien hypertexte vers l'article, en utilisant le package hyperref \LaTeX{}.

\section*{Remerciements}

This document has been adapted
by Steven Bethard, Ryan Cotterell and Rui Yan
from the instructions for earlier ACL and NAACL proceedings, including those for 
ACL 2019 by Douwe Kiela and Ivan Vuli\'{c},
NAACL 2019 by Stephanie Lukin and Alla Roskovskaya, 
ACL 2018 by Shay Cohen, Kevin Gimpel, and Wei Lu, 
NAACL 2018 by Margaret Mitchell and Stephanie Lukin,
Bib\TeX{} suggestions for (NA)ACL 2017/2018 from Jason Eisner,
ACL 2017 by Dan Gildea and Min-Yen Kan, 
NAACL 2017 by Margaret Mitchell, 
ACL 2012 by Maggie Li and Michael White, 
ACL 2010 by Jing-Shin Chang and Philipp Koehn, 
ACL 2008 by Johanna D. Moore, Simone Teufel, James Allan, and Sadaoki Furui, 
ACL 2005 by Hwee Tou Ng and Kemal Oflazer, 
ACL 2002 by Eugene Charniak and Dekang Lin, 
and earlier ACL and EACL formats written by several people, including
John Chen, Henry S. Thompson and Donald Walker.
Additional elements were taken from the formatting instructions of the \emph{International Joint Conference on Artificial Intelligence} and the \emph{Conference on Computer Vision and Pattern Recognition}.

% Entries for the entire Anthology, followed by custom entries
\bibliography{bibliographie.bib}

\appendix

\section{Exemple d'Annexe}
\label{sec:appendix}

Ceci est une annexe.

\end{document}
