% je choisis le type de document que je veux écrire: article, book, beamer (pour un équivalent de powerpoint). C'est ici que je peux préciser que je vais passer sur deux colonnes, utiliser de l'a4, la taille de ma police…
\documentclass[paper=a4,
               fontsize=11pt,
               twoside, %rectoverso
               %twocolumn,
              ]{article}

% Nous allons désormais ajouter des packages offrant des options supplémentaires, comme en python (import) ou en R (library())

%je dis que mon document va être en français
\usepackage[french]{babel}

% Si vous choisissez "XeLaTeX" comme typesetting engine, je vous recommande l'utilisation de fontpsec préventivement pour gérer des caractères spéciaux
\usepackage{fontspec}
% si vous restez sur pdfLaTeX pour ajouter les caractères utf-8 aux traditionnels ASCII
%\usepackage[utf8]{inputenc}



%le package caption va me permettre mettre des illustrations
\usepackage{graphicx}

%ce package va me permettre de mettre du code dans mon article
\usepackage{minted}

% ici nous allons préciser ce que nous voulons pour la bibliographie
\usepackage[backend=biber,%on utilise biber comme backend
            style=authoryear-comp,%je veux l'auteur puis la date
            sorting=ynt %la biblio est organisée par year, name, title
            ]{biblatex}
%nous précisons où se trouve la bibliographie
\addbibresource{bibliography.bib}

%ici vous remplissez à l'avance des informations qui vont servir à remplir la titraille de votre article. Ces informations sont réutilisés par l \maketitle plus bas
\title{Cours latex}
\author{Simon Gabay}
\date{May 2022}
% il est possible de laisser \date{} vide pour ne pas mettre de date

% Le document que nous voulons écrire commence ici
\begin{document}
    
    %j'appelle les informations de titre
    \maketitle
    
    Je commence une introduction \textit{avec de l'italique} (la commande \texttt{\textbackslash{}emph} produit le même effet: \emph{exemple}), un \textbf{peu de gras}, des \textsc{petites capitales} et combiner ces options \textbf{\textit{tout en même temps}}.
    On note qu'un retour à la ligne dans le code n'implique pas un retour à la ligne dans \LaTeX 
    
    Mais deux retours à la ligne font bien un nouveau paragraphe. Je peux aussi mettre des notes de bas de page \footnote{le texte se retrouve donc tout en bas de la page}.
    
    % je peux faire des parties
    \section{Introduction}

    % je peux rajouter une image
    %[!htp] précise que je veux l'image là où elle apparait (here) ou en haut (top) ou en bas (bottom), dans cet ordre
    \begin{figure}[!htb]
        %je centre l'image
        \centering
        %je précise la taille de mon image et le chemin vers elle
        \includegraphics[height=2cm]{images/logoLatex.png}
        %je rajoute une légende
        \caption{ici un logo}
        %je lui donne un identifiant pour pointer vers l'image par la suite
        \label{fig:logolatex}
    \end{figure}
    
     Je peux citer un article \cite{baptiste_transferring_2021}, mettre des parenthèses autour de la citation \parencite{baptiste_transferring_2021} ou la mettre en note \footcite{baptiste_transferring_2021}. Je peux aussi renvoyer à l'image (cf. image~\ref{fig:logolatex}).
     
    %des sous-parties
    \subsection{Petit A}
    %je peux rajouter des label un peu partout où je pourrais avoir besoin de pointer, comme une sous-partie à laquelle je ferais référence plus loin dans mon texte
    \label{petitA}
    Rajouter des tableaux est un peu plus compliqué:
    
    \begin{table}[!htp]
        \centering
        % je commence le tableau avec de colonnes (c) séparées par une barre verticale puis une troisième
        \begin{tabular}{c|cc}
            En-tête 1 & En-tête 2 & En-tête 3 \\
            %je rajoute une barre horizontale
        \hline
            %on note que que je rajoute \\ pour signaler une nouvelle ligne
            U & V & W \\
            X & Y & Z \\
        \end{tabular}
        \caption{mon premier tableau}
        \label{tab:tableau}
    \end{table}
    
    Je peux faire des listes
    \begin{itemize}
        \item un
        \item deux
        \item trois
    \end{itemize}

    Je peux faire des listes numérotées:
    \begin{enumerate}
        \item un
        \item deux
        \item trois
    \end{enumerate}
    
    Je peux mettre une liste dans une liste
    \begin{enumerate}
        \item un
        \begin{itemize}
            \item deux
            \item trois
            \item quatre
        \end{itemize}
        \item cinq
        \item six
    \end{enumerate}    

    Pour mettre du code, j'utilise \texttt{minted}:
    \begin{minted}{python}
import panda as pd
    \end{minted}
    \begin{minted}{xml}
là je mets du <balise attribut="valeur">xml</balise>
    \end{minted}
    
    Je peux mettre du texte en \textsuperscript{exposant} comme pour \textsc{xvii}\textsuperscript{e}~s.
    
    Pour les guillemets français j'utilise des choses un peu \og~bizarres~\fg{}
    
    Comme le français, et d'autres langues, utilisent des caractères particuliers absents en anglais, il est possible de les rajouter comme avec: \`A
    
    
    %je passe à la page suivante:
    \clearpage
    
    %J'ajoute la bibliographie
    \printbibliography

\end{document}
